\documentclass[onecolumn,prd,nofootinbib]{revtex4-1}
\usepackage{graphicx}
\usepackage{amsmath}
\usepackage{amssymb}
\usepackage{bm}
\usepackage{color}
\usepackage{verbatim}


% shortcuts
%\newcommand{\ud}{\,\mathrm{d}}
\newcommand{\bD}{\boldsymbol D}
\newcommand{\bC}{\boldsymbol C}
\newcommand{\bDelta}{\boldsymbol \Delta}
\newcommand{\Dgal}{D^{\rm gal}}
\newcommand{\Dne}{D^{\bar{n}_e}}
\newcommand{\Dde}{D^{\delta_e}}
\newcommand{\Dsrc}{D^{\rm src}}
\newcommand{\del}{\delta\!}
\newcommand{\calP}{{\cal P}}
\newcommand{\ud}{\,\mathrm{d}}



% for draft comments.
\newcommand{\red}{\textcolor{red}}
\newcommand{\blue}{\textcolor{blue}}
\newcommand{\green}{\textcolor{green}}


\begin{document}

\title{The large scale clustering of fast radio bursts in dispersion measure space}

\author{Kiyoshi Wesley Masui}
 \affiliation{Department of Physics and Astronomy, University of British 
Columbia, Vancouver, BC, Canada, V6T 1Z1}
 \affiliation{Canadian Institute for Advanced Research, CIFAR Program in
Cosmology and Gravity, Toronto, ON, Canada, M5G 1Z8}

\author{Kris Sigurdson}
 \affiliation{Department of Physics and Astronomy, University of British 
Columbia, Vancouver, BC, Canada, V6T 1Z1}


\begin{abstract}

\end{abstract}

\maketitle

\section{FRB density perturnbations in DM space}

\begin{equation}
D(\hat n, \chi) = \int_0^\chi \ud \chi' a(\chi')^2
   \left[1 + \delta_e(\hat{n}\chi')\right]
\end{equation}

This defines DM space:
\begin{equation}
D(\chi_s) =  \int_0^{\chi_s} \ud \chi' a(\chi')^2
\end{equation}

Combining the above, we find that:
\begin{equation}
\frac{\ud \chi_s}{\ud \chi} = (1 + \delta_e),
\end{equation}
and
\begin{equation}
\chi_s - \chi = \int_0^\chi \ud \chi' \delta_e(\hat n \chi').
\end{equation}

\begin{equation}
\label{e:density}
n_{fs}(\vec x_s) \ud^3\vec x_s = n_{f}(\vec x) \ud^3\vec x
\end{equation}

Averaged over the sky, the galaxy density should be the same in DM 
space as in real space.
\begin{equation}
\bar{n}_{fs}(\chi_s)\left[ 1 + \delta_f(\vec x_s)\right] \ud^3\vec x_s
    = \bar{n}_{f}(\chi)\left[ 1 + \delta_f(\vec x)\right] \ud^3\vec x
\end{equation}

\begin{equation}
\bar{n}_{fs}(\chi) = \bar{n}_f(\chi)
\end{equation}
Therefore,
\begin{align}
\bar{n}_{fs}(\chi_s) 
    &= \bar{n}_{f}(\chi) + (\chi_s - \chi)\frac{\ud \bar{n}_f}{\ud \chi}\\
    &= \bar{n}_{f}(\chi)
       + \frac{\ud \bar{n}_f}{\ud \chi}\int_0^\chi \ud \chi' \delta_e(\hat n \chi')
       \label{e:nfs}
\end{align}

The jacobian in spherical coordinates is:
\begin{align}
\left| \frac{\ud^3\vec x}{\ud^3\vec x_s} \right|
    &= \frac{\ud \chi}{\ud \chi_s}\frac{\chi^2}{\chi_s^2}\\
    &= (1 + \delta_e)^{-1}
       \left(1 + \frac{\int_0^\chi \ud \chi' \delta_e(\hat n \chi')}
                      {\chi}\right)^{-2}\\
    &\approx 1 - \delta_e
        - \frac{2}{\chi}\int_0^\chi \ud \chi' \delta_e(\hat n \chi')
        \label{e:jac}
\end{align}
Following Kaiser, the third line follows since the
second factor in parenthases on the second line, which is the
average overdensity along the line of sight, is much
closer to unity than the first due to cancelations.

Substituting Equations~\ref{e:nfs}~and~\ref{e:jac} into Equation~\ref{e:density},
we obtain:
\begin{equation}
\label{e:delta_s}
    \delta_{fs} = \delta_f - \delta_e
    - \left(\frac{1}{\bar{n}_f}\frac{\ud \bar{n}_f}{\ud \chi}
    + \frac{2}{\chi} \right)
        \int_0^\chi \ud \chi' \delta_e(\hat n \chi')
\end{equation}

\section{Two-point statistics in DM space}

In this section all quantities are already first order in the perturbations so
there is no need to distinguish between the DM space coordinates and the real
space coordintes. The correction is higher order. Still need to distinguish
between DM space and real space fields ($\delta_{fs} \neq \delta_f$, but
$\delta_f(\vec x_s) \approx \delta_f(\vec x)$).

Unlike in redshift-space distortions, Equation~\ref{e:delta_s} does not have a
simple form in harmonic space. The equations third term couples harmonic modes
and thus the two-point statistic cannot be phrased as a simple power spectrum.
We will phrase the two point statistic in terms of $C^{ss}_{\ell D D'}$, which is
the cross-correlation angular power-spectrum of the DM space source overdensity
on DM shells at $D$ and $D'$.

Because the expression for the density perturbations in DM space involves an
integral over the line of sight, which for far off sources may cover a
significant amount of the Universe's expansion, it is nessisary to explicitly
account for light cone effects. This is most easily done in linear theory,
where these effects are scale independant, a factorization that allows certian
integral to be performed. In reality, non-linear growth causes both the bias
and the growth factor to be scale and time dependant. As in weak lensing, this
would need to be simulated.

We use the following convensions:
\begin{equation}
    P_X(k, \chi) = b_X^2 g(\chi)^2 a(\chi)^2 P(k, 0)
\end{equation}

\begin{equation}
    \delta_X(\vec x, \chi) =  b_X g(\chi) a(\chi) \int\frac{\ud^3\vec k}{(2 \pi)^3}
        e^{i\vec k \cdot \vec x} \delta(\vec k, 0)
\end{equation}

\begin{equation}
    w^{XX}(\hat n \cdot \hat n', \chi, \chi')
    \equiv \langle \delta_X(\vec x) \delta_X(\vec x') \rangle
\end{equation}

For dark matter:
\begin{align}
w(\hat n \cdot \hat n', \chi, \chi') 
    &= gg'aa'
        \int\frac{\ud^3\vec k}{(2 \pi)^3} \frac{\ud^3\vec k'}{(2 \pi)^3}
        e^{i\vec k \cdot \vec x} e^{i\vec k' \cdot \vec x'}
        \langle \delta(\vec k, 0) \delta(\vec k', 0) \rangle\\
    &= gg'aa'
        \int\frac{\ud^3\vec k}{(2 \pi)^3} 
        e^{i\vec k \cdot (\vec x - \vec x')} P(k, 0)\\
    &= gg'aa'
        \int\frac{\ud^3\vec k}{(2 \pi)^3} 
        e^{i\vec k \cdot (\vec x - \vec x')} P(k)\\
    &= gg'aa'
        \int\frac{k^2 \sin(\theta_k) \ud k \ud \theta_k \ud \phi_k}{(2 \pi)^3} 
        e^{i k |\vec x - \vec x'| \cos\theta_k} P(k, 0)\\
    &= gg'aa'
        \int_0^\infty\frac{\ud k}{2 \pi^2} k^2
        \frac{\sin(k |\vec x - \vec x'|)}{k |\vec x - \vec x'|} P(k, 0)\\
    &= gg'aa'
        \int_0^\infty\frac{\ud k}{2 \pi^2} k^2
        j_0 (k \sqrt{\chi^2 + \chi'^{^2} - \chi \chi' \hat n \cdot \hat n'})
        P(k, 0)
\end{align}
Somehow when I integrate this against a lengendre polynomial, $P_\ell(\mu)$,
to get $C_\ell$
I'm going to get Bessel functions. Lets assume that works (Kris, how does this
work?) to get:
\begin{equation}
    C_{\ell\chi\chi'} = \frac{2 gg'aa'}{\pi}
    \int_0^\infty\ud k k^2 j_\ell(k \chi) j_\ell(k \chi')
        P(k, 0)
\end{equation}

Define $\delta_d$ as:
\begin{equation}
    \delta_d(\hat n \chi) = \int_0^\chi \ud \chi' \delta_e(\hat n \chi')
\end{equation}
which doesn't have the same units as overdensity, but that is okay.
\begin{align}
    w^{dd}(\hat n \cdot \hat n', \chi, \chi')
    &= \int_0^\chi \ud \chi'' \int_0^{\chi'} \ud \chi'''
         \langle \delta_e(\hat n \chi'') \delta_e(\hat n' \chi''') \rangle\\
    &= b_e^2\int_0^\chi \ud \chi'' \int_0^{\chi'} \ud \chi'''
        g''g'''a''a'''
        \int\frac{\ud^3\vec k}{(2 \pi)^3} \frac{\ud^3\vec k'}{(2 \pi)^3}
        e^{i\vec k \cdot \vec x''} e^{i\vec k' \cdot \vec x'''}
        \langle \delta(\vec k, 0) \delta(\vec k', 0) \rangle\\
    &= b_e^2\int_0^\chi \ud \chi'' \int_0^{\chi'} \ud \chi'''
        g''g'''a''a'''
        \int_0^\infty\frac{\ud k}{2 \pi^2} k^2
        j_0 (k \sqrt{\chi''^2 + \chi'''^{^2} - \chi'' \chi''' \hat n \cdot \hat n'})
        P(k, 0)
\end{align}




\end{document}
