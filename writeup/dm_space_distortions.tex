\documentclass[onecolumn,prd,nofootinbib]{revtex4-1}
\usepackage{graphicx}
\usepackage{amsmath}
\usepackage{amssymb}
\usepackage{bm}
\usepackage{color}
\usepackage{verbatim}


% shortcuts
%\newcommand{\ud}{\,\mathrm{d}}
\newcommand{\bD}{\boldsymbol D}
\newcommand{\bC}{\boldsymbol C}
\newcommand{\bDelta}{\boldsymbol \Delta}
\newcommand{\Dgal}{D^{\rm gal}}
\newcommand{\Dne}{D^{\bar{n}_e}}
\newcommand{\Dde}{D^{\delta_e}}
\newcommand{\Dsrc}{D^{\rm src}}
\newcommand{\del}{\delta\!}
\newcommand{\calP}{{\cal P}}
\newcommand{\ud}{\,\mathrm{d}}



% for draft comments.
\newcommand{\red}{\textcolor{red}}
\newcommand{\blue}{\textcolor{blue}}
\newcommand{\green}{\textcolor{green}}


\begin{document}

\title{The large scale clustering of fast radio bursts in dispersion measure space}

\author{Kiyoshi Wesley Masui}
 \affiliation{Department of Physics and Astronomy, University of British 
Columbia, Vancouver, BC, Canada, V6T 1Z1}
 \affiliation{Canadian Institute for Advanced Research, CIFAR Program in
Cosmology and Gravity, Toronto, ON, Canada, M5G 1Z8}

\author{Kris Sigurdson}
 \affiliation{Department of Physics and Astronomy, University of British 
Columbia, Vancouver, BC, Canada, V6T 1Z1}


\begin{abstract}

\end{abstract}

\maketitle

\section{FRB density perturbations in DM space}

\begin{equation}
D(\hat n, \chi) = \int_0^\chi \frac{\ud\chi'}{a(\chi')}
   \left[1 + \delta_e(\hat{n}\chi')\right]
\end{equation}

This defines DM space:
\begin{equation}
D(\chi_s) = \int_0^{\chi_s}\frac{\ud\chi'}{a(\chi')}
\end{equation}

Combining the above, we find that:
\begin{equation}
\frac{\ud \chi_s}{\ud \chi} = (1 + \delta_e),
\end{equation}
and
\begin{equation}
\chi_s - \chi = \int_0^\chi \ud \chi' \delta_e(\hat n \chi').
\end{equation}

\begin{equation}
\label{e:density}
n_{fs}(\vec x_s) \ud^3\vec x_s = n_{f}(\vec x) \ud^3\vec x
\end{equation}

Averaged over the sky, the galaxy density should be the same in DM 
space as in real space.
\begin{equation}
\bar{n}_{fs}(\chi_s)\left[ 1 + \delta_f(\vec x_s)\right] \ud^3\vec x_s
    = \bar{n}_{f}(\chi)\left[ 1 + \delta_f(\vec x)\right] \ud^3\vec x
\end{equation}

\begin{equation}
\bar{n}_{fs}(\chi) = \bar{n}_f(\chi)
\end{equation}
Therefore,
\begin{align}
\bar{n}_{fs}(\chi_s) 
    &= \bar{n}_{f}(\chi) + (\chi_s - \chi)\frac{\ud \bar{n}_f}{\ud \chi}\\
    &= \bar{n}_{f}(\chi)
       + \frac{\ud \bar{n}_f}{\ud \chi}\int_0^\chi \ud \chi' \delta_e(\hat n \chi')
       \label{e:nfs}
\end{align}

The jacobian in spherical coordinates is:
\begin{align}
\left| \frac{\ud^3\vec x}{\ud^3\vec x_s} \right|
    &= \frac{\ud \chi}{\ud \chi_s}\frac{\chi^2}{\chi_s^2}\\
    &= (1 + \delta_e)^{-1}
       \left(1 + \frac{\int_0^\chi \ud \chi' \delta_e(\hat n \chi')}
                      {\chi}\right)^{-2}\\
    &\approx 1 - \delta_e
        - \frac{2}{\chi}\int_0^\chi \ud \chi' \delta_e(\hat n \chi')
        \label{e:jac}
\end{align}
Following Kaiser, the third line follows since the
second factor in parentheses on the second line, which is the
average overdensity along the line of sight, is much
closer to unity than the first due to cancellations.

Substituting Equations~\ref{e:nfs}~and~\ref{e:jac} into Equation~\ref{e:density},
we obtain:
\begin{equation}
\label{e:delta_s}
    \delta_{fs} = \delta_f - \delta_e
    - \left(\frac{1}{\bar{n}_f}\frac{\ud \bar{n}_f}{\ud \chi}
    + \frac{2}{\chi} \right)
        \int_0^\chi \ud \chi' \delta_e(\hat n \chi')
\end{equation}

\section{Two-point statistics in DM space}

In this section all quantities are already first order in the perturbations so
there is no need to distinguish between the DM space coordinates and the real
space coordinates. The correction is higher order. Still need to distinguish
between DM space and real space fields ($\delta_{fs} \neq \delta_f$, but
$\delta_f(\vec x_s) \approx \delta_f(\vec x)$).

Unlike in redshift-space distortions, Equation~\ref{e:delta_s} does not have a
simple form in harmonic space. The equations third term couples harmonic modes
and thus the two-point statistic cannot be phrased as a simple power spectrum.
We will phrase the two point statistic in terms of $C^{ss}_{\ell D D'}$, which is
the cross-correlation angular power-spectrum of the DM space source overdensity
on DM shells at $D$ and $D'$.

Because the expression for the density perturbations in DM space involves an
integral over the line of sight, which for far off sources may cover a
significant amount of the Universe's expansion, it is necessary to explicitly
account for light cone effects. This is most easily done in linear theory,
where these effects are scale independent, a factorization that allows certain
integral to be performed. In reality, non-linear growth causes both the bias
and the growth factor to be scale and time dependant. As in weak lensing, this
would need to be simulated.

We use the following conventions:
\begin{equation}
    P_{xy}(k, \chi) = b_x b_y g(\chi)^2 a(\chi)^2 P(k, 0)
\end{equation}

\begin{equation}
    \delta_x(\vec x, \chi) =  b_x g(\chi) a(\chi) \int\frac{\ud^3\vec k}{(2 \pi)^3}
        e^{i\vec k \cdot \vec x} \delta(\vec k, 0)
\end{equation}

\begin{equation}
    w^{xy}(\hat n \cdot \hat n', \chi, \chi')
    \equiv \langle \delta_x(\vec x) \delta_y(\vec x') \rangle
\end{equation}

For any statistically stationary tracers $x$ and $y$,
ignoring time dependance (redshift evolution)
of the power spectrum:
\begin{align}
    \delta_{\ell\ell'}\delta_{mm'}C^{xy}_\ell(\chi, \chi')
    &= \int\ud\Omega\ud\Omega'Y_{\ell m}(\hat n) Y_{\ell' m'}(\hat n')
        \langle \delta^x(\hat n \chi) \delta^y(\hat n' \chi') \rangle
        \\
    &=
        \int\ud\Omega\ud\Omega'Y_{\ell m}(\hat n) Y_{\ell' m'}(\hat n')
        \int\frac{\ud^3\vec k}{(2 \pi)^3} \frac{\ud^3\vec k'}{(2 \pi)^3}
        e^{i\vec k \cdot \vec x} e^{i\vec k' \cdot \vec x'}
        \langle \delta(\vec k) \delta(\vec k') \rangle
        \\
    &=
        \int\ud\Omega\ud\Omega'Y_{\ell m}(\hat n) Y_{\ell' m'}(\hat n')
        \int\frac{\ud^3\vec k}{(2 \pi)^3} 
        e^{i\vec k \cdot (\vec x - \vec x')} P_{xy}(k)
        \\
    &=
        \int\frac{\ud^3\vec k}{(2 \pi)^3} P_{xy}(k)
        \int\ud\Omega Y_{\ell m}(\hat n)
        e^{i\vec k \cdot \vec x}
        \int\ud\Omega' Y_{\ell' m'}(\hat n')
        e^{-i\vec k \cdot \vec x'}
        \\
    &=
        \int\frac{\ud^3\vec k}{(2 \pi)^3} P_{xy}(k)
        \left[(4 \pi) i^\ell j_\ell(k\chi) Y_{\ell m}^*(\hat k)\right]
        \left[(4 \pi) i^{-\ell'} j_{\ell'}(k\chi') Y_{\ell' m'}(\hat
        k)\right]
        \\
    &=
        \frac{2}{\pi}
        \int_0^\infty\ud k k^2 j_\ell(k\chi) j_{\ell'}(k\chi')P_{xy}(k)
        \int\ud \Omega_k i^{(\ell - \ell')}
        Y_{\ell' m'}^*(\hat k) Y_{\ell m}(\hat k)
        \\
    &= \delta_{\ell\ell'}\delta_{mm'}\frac{2}{\pi}
        \int_0^\infty\ud k k^2  j_\ell(k\chi) j_{\ell}(k\chi')P_{xy}(k)
        \\
    C^{xy}_\ell(\chi,\chi') 
    &= \frac{2}{\pi}
\int_0^\infty\ud k k^2 j_\ell(k\chi) j_{\ell}(k\chi')P_{xy}(k)
\end{align}

It is convienient to reparameterize this in terms of
$\bar\chi \equiv (\chi + \chi') /2$ and $\Delta \chi \equiv \chi' - \chi$.
We expect the power spectrum to vary slowly as a function of $\bar\chi$ and
very quickly as a function of $\Delta\chi$. However it will be highly
suppressed for $\ell \gg \bar\chi/\Delta\chi$. These facts will be very
helpfull in subsequent numerical integrals.

Also since smallish value of $\Delta\chi$ are of interest, this leads us to
an obviouse way to include redshift dependance of the
power spectrum:
\begin{equation}
C^{xy}_\ell(\bar\chi,\Delta\chi) 
    = \frac{2}{\pi}
    \int_0^\infty\ud k k^2
    j_\ell(k\chi) j_{\ell}(k\chi')
    P_{xy}(k,\bar\chi) \qquad \Delta\chi \ll 1/aH
\end{equation}

Define $\delta_d$ as:
\begin{equation}
    \delta_d(\hat n \chi) = \int_0^\chi \ud \chi' \delta_e(\hat n \chi')
\end{equation}
which doesn't have the same units as over-density, but that is okay.
\begin{equation}
C^{dd}_\ell(\chi,\chi')
    =
    \frac{2}{\pi}
    \int_0^\chi\ud\chi''
    \int_0^{\chi'}\ud\chi'''
    \int_0^\infty\ud k k^2 j_\ell(k\chi'') j_{\ell}(k\chi'''
    )P_{ee}(k, (\chi''+\chi''')/2)
\end{equation}

For numerical integration, perform the following change of variables to make
things much easier.
\begin{align}
C^{dd}_\ell(\chi,\chi')
&=
    \frac{2}{\pi}
    \int_0^{(\chi' + \chi)/2}\ud\bar\chi'
    \int_{\max(-2\bar\chi', -2(\chi -\bar\chi'))}^{\min(2\bar\chi', 2(\chi' -
    \bar\chi'))}\ud\Delta'
    \int_0^\infty\ud k k^2 j_\ell(k(\bar\chi' - \Delta'/2))
    j_{\ell}(k(\bar\chi' + \Delta'/2))
    P_{ee}(k, \bar\chi')
    \\
    &\approx
    \frac{4}{\pi}
    \int_0^{\chi_m}\ud\bar\chi'
    \int_{0}^{\infty}\ud\Delta'
    \int_0^\infty\ud k k^2 j_\ell(k(\bar\chi' - \Delta'/2))
    j_{\ell}(k(\bar\chi' + \Delta'/2))
    P_{ee}(k, \bar\chi')
\end{align}
At this point we could again to the transformation to $C_{\ell\chi\chi'}$
leaving the expression as a triple integral. This is not as computationally
costly as might be expected at first glance, since the $\chi''$ and $\chi'''$
integrals are almost the same for all $\chi$, $\chi'$ pairs with only the
limits of integration changing. Thus the full set of $C_{\ell\chi\chi'}$'s can
be computed with a cumulative sum, only performing one new integral (over $k$)
for each element.

Finally, the cross terms:
\begin{equation}
C^{dx}_\ell(\chi,\chi')
    =
    \frac{2}{\pi}
    \int_0^\chi\ud\chi''
    \int_0^\infty\ud k k^2 
    j_\ell(k\chi') j_{\ell}(k\chi'')P_{ex}(k, (\chi' + \chi'')/2)
\end{equation}

with the following change of variables for numerical integration:
\begin{align}
C^{dx}_\ell(\chi,\chi')
    =&
    \frac{2}{\pi}
    \int_{-\chi'}^{\chi - \chi'}\ud\Delta'
    \int_0^\infty\ud k k^2 j_\ell(k\chi')
    j_{\ell}(k(\chi'+\Delta'))
    P_{ex}(k, \chi' + \Delta'/2)
    \\
C^{dx}_\ell(\chi,\chi') + C^{xd}_\ell(\chi,\chi')
    \approx&
    \frac{2}{\pi}
    \int_{-\chi'}^{\chi - \chi'}\ud\Delta'
    \int_0^\infty\ud k k^2 j_\ell(k(\chi'-\Delta'/2))
    j_{\ell}(k(\chi'+\Delta'/2))
    P_{ex}(k, \chi')
    \nonumber\\
    &+
    \frac{2}{\pi}
    \int_{-\chi}^{\chi' - \chi}\ud\Delta'
    \int_0^\infty\ud k k^2 j_\ell(k(\chi-\Delta'/2))
    j_{\ell}(k(\chi+\Delta'/2))
    P_{ex}(k, \chi)\\
    \approx&
    \frac{4}{\pi}
    \int_{0}^{\infty}\ud\Delta'
    \int_0^\infty\ud k k^2 j_\ell(k(\chi_m-\Delta'/2))
    j_{\ell}(k(\chi_m+\Delta'/2))
    P_{ex}(k, \chi_m)
\end{align}
The second line exploits the fact that for $\ell \gg 1$, $\chi \gg \chi'/\ell$ and
$\chi' \gg \chi/\ell$, only $\Delta'\ll\chi$ contributes significantly to the
integral. Also the expressions are very weak functions of $\chi$ and thus
may be shifted by $\Delta/2$ without incurring much error. The third line
exploits the fact that only one term contributes if $\chi$ and $\chi'$ are
widely separated, and if they are close then the two terms combine to the given
expression.

\red{Actually, with the approximations we've made, we can actually do the
$\Delta$ integrals analytically by transferring all the $\Delta$ to a single
Bessel and shifting the power spectrum to be independent of $\Delta$.  Should
be pretty good for $\ell > 10$.  That is nice because it will control all our
numerical issues with oscillating Bessel functions.}


\section{Small angles and Limber}

Alternately, on the flat sky and for small angles:
\begin{align}
(2\pi)^2\delta^2(\vec \ell - \vec \ell')C_\ell(\chi, \chi')
    &=
        \int\ud^2\vec\theta\ud^2\vec\theta'
        e^{-i\vec\ell\cdot\vec\theta} e^{-i\vec\ell'\cdot\vec\theta'}
        \langle \delta(\hat n \chi) \delta(\hat n' \chi') \rangle
        \\
    &= gg'aa'
        \int\ud^2\vec\theta\ud^2\vec\theta'
        e^{-i\vec\ell\cdot\vec\theta} e^{-i\vec\ell'\cdot\vec\theta'}
        \int\frac{\ud^3\vec k}{(2 \pi)^3} 
        e^{i\vec k \cdot (\vec x - \vec x')} P(k, 0)
        \\
    &= gg'aa'
        \int\frac{\ud^3\vec k}{(2 \pi)^3} 
        e^{i k_\parallel (\chi - \chi')} P(k, 0)
        \int\ud^2\vec\theta e^{-i(\vec\ell-\bar\chi\vec k_\bot)\cdot\vec\theta}
        \int\ud^2\vec\theta' e^{-i(\vec\ell'-\bar\chi\vec k_\bot)\cdot\vec\theta'}
        \\
    &= gg'aa'
        \int\frac{\ud^3\vec k}{(2 \pi)^3} 
        e^{i k_\parallel (\chi - \chi')} P(\sqrt{k_\parallel^2 + k_\bot^2}, 0)
        (2\pi)^4 \delta^2(\vec\ell - \bar\chi \vec k_\bot)
        \delta^2(\vec\ell' - \bar\chi\vec k_\bot)
        \\
    &= gg'aa'
        \int_{-\infty}^\infty\frac{\ud k_\parallel}{(2 \pi)} 
        e^{i k_\parallel (\chi - \chi')}
        P(\sqrt{k_\parallel^2 + \ell^2/\bar\chi^2}, 0)
        \frac{(2\pi)^2\delta^2(\vec\ell' - \vec \ell)}{\bar\chi^2}
        \\
C_\ell(\chi, \chi')
    &= \frac{gg'aa'}{\bar\chi^2}
        \int_{-\infty}^\infty\frac{\ud k_\parallel}{(2 \pi)} 
        e^{i k_\parallel (\chi - \chi')}
        P(\sqrt{k_\parallel^2 + \ell^2/\bar\chi^2}, 0)
\end{align}
Which is valid for $\ell \gg 1$ and $\chi - \chi' \ll \bar\chi$.
$\bar \chi \equiv (\chi + \chi') /2$


Or for flat sky and small angles:
\begin{align}
C^{dd}_\ell(\chi,\chi') 
    &=
    \int_0^\chi\ud\chi''g''a''
    \int_0^{\chi'}\ud\chi'''g'''a'''
    \frac{1}{\bar\chi^2}
    \int_{-\infty}^\infty\frac{\ud k_\parallel}{(2 \pi)} 
    e^{i k_\parallel (\chi'' - \chi''')}
    P_{ee}(\sqrt{k_\parallel^2 + \ell^2/\bar\chi^2}, 0)
    \\
    &\approx
    \int_0^\chi\ud\chi''g''a''
    \int_0^{\chi'}\ud\chi'''g'''a'''
    \frac{1}{\bar\chi^2}
    \int_{-\infty}^\infty\frac{\ud k_\parallel}{(2 \pi)} 
    e^{i k_\parallel (\chi'' - \chi''')}
    P_{ee}(\ell/\bar\chi, 0)
    \\
    &=
    \int_0^\chi\ud\chi''g''a''
    \int_0^{\chi'}\ud\chi'''g'''a'''
    \frac{1}{\bar\chi^2}
    \delta(\chi'' - \chi''')
    P_{ee}(\ell/\bar\chi, 0)
    \\
    &=
    \int_0^{\min(\chi,\chi')}\ud\chi''
    \left(\frac{g''a''}{\chi''}\right)^2
    P_{ee}(\ell/\chi'', 0)
\end{align}
\red{Double check the $\min$.}
where the second line applies Limber's approximation.

Or under Limber:
\begin{align}
C^{dx}_\ell(\chi,\chi')
    &=
    g'a'
    \int_0^\chi\ud\chi''g''a''
    \frac{1}{\bar\chi^2}
    \delta(\chi'' - \chi')
    P_{ex}(\ell/\bar\chi, 0)
    \\
    &=
    \left(\frac{g'a'}{\chi'}\right)^2
    P_{ex}(\ell/\chi', 0)
    \qquad \textrm{if $\chi' < \chi$ and 0 otherwise.}
    \\
C^{dx}_\ell(\chi,\chi') + C^{xd}_\ell(\chi,\chi')
    &=
    \left(\frac{g^ma^m}{\chi^m}\right)^2
    P_{ex}(\ell/\chi^m, 0)
\end{align}

Accumulating all terms for Limber and assuming a linear bias model:
\begin{align}
C^{ss}_\ell(\chi_m,\Delta\chi) =&~
    (b_f - b_e)^2 \left(\frac{g_m a_m}{\chi_m}\right)^2
        \int_{-\infty}^\infty\frac{\ud k_\parallel}{(2 \pi)} 
        e^{i k_\parallel \Delta\chi}
        P(\sqrt{k_\parallel^2 + \ell^2/\chi_m^2}, 0)
    \\ &+
    \left(\frac{1}{\bar{n}_f}\frac{\ud \bar{n}_f}{\ud \chi}
    + \frac{2}{\chi} \right)_{\chi_m}^2
    b_e^2 \int_0^{\chi_m}\ud\chi''
        \left(\frac{g''a''}{\chi''}\right)^2
        P(\ell/\chi'', 0)
    \\ &+
    \left(\frac{1}{\bar{n}_f}\frac{\ud \bar{n}_f}{\ud \chi}
    + \frac{2}{\chi} \right)_{\chi_m}
    b_e(b_f - b_e)\left(\frac{g_m a_m}{\chi_m}\right)^2
        P(\ell/\chi_m, 0)
\end{align}

%%%% SCRAP

\begin{comment}
\begin{align}
w(\hat n \cdot \hat n', \chi, \chi') 
    &= gg'aa'
        \int\frac{\ud^3\vec k}{(2 \pi)^3} \frac{\ud^3\vec k'}{(2 \pi)^3}
        e^{i\vec k \cdot \vec x} e^{i\vec k' \cdot \vec x'}
        \langle \delta(\vec k, 0) \delta(\vec k', 0) \rangle\\
    &= gg'aa'
        \int\frac{\ud^3\vec k}{(2 \pi)^3} 
        e^{i\vec k \cdot (\vec x - \vec x')} P(k, 0)\\
    &= gg'aa'
        \int\frac{\ud^3\vec k}{(2 \pi)^3} 
        e^{i\vec k \cdot (\vec x - \vec x')} P(k)\\
    &= gg'aa'
        \int\frac{k^2 \sin(\theta_k) \ud k \ud \theta_k \ud \phi_k}{(2 \pi)^3} 
        e^{i k |\vec x - \vec x'| \cos\theta_k} P(k, 0)\\
    &= gg'aa'
        \int_0^\infty\frac{\ud k}{2 \pi^2} k^2
        \frac{\sin(k |\vec x - \vec x'|)}{k |\vec x - \vec x'|} P(k, 0)\\
    &= gg'aa'
        \int_0^\infty\frac{\ud k}{2 \pi^2} k^2
        j_0 (k \sqrt{\chi^2 + \chi'^{^2} - \chi \chi' \hat n \cdot \hat n'})
        P(k, 0)
\end{align}
Somehow when I integrate this against a Legendre polynomial, $P_\ell(\mu)$,
to get $C_\ell$
I'm going to get Bessel functions. Lets assume that works (Kris, how does this
work?) to get:
\begin{equation}
    C_{\ell\chi\chi'} = \frac{2 gg'aa'}{\pi}
    \int_0^\infty\ud k k^2 j_\ell(k \chi) j_\ell(k \chi')
        P(k, 0)
\end{equation}
\end{comment}


\begin{comment}
\begin{align}
    w^{dd}(\hat n \cdot \hat n', \chi, \chi')
    &= \int_0^\chi \ud \chi'' \int_0^{\chi'} \ud \chi'''
         \langle \delta_e(\hat n \chi'') \delta_e(\hat n' \chi''') \rangle\\
    &= b_e^2\int_0^\chi \ud \chi'' \int_0^{\chi'} \ud \chi'''
        g''g'''a''a'''
        \int\frac{\ud^3\vec k}{(2 \pi)^3} \frac{\ud^3\vec k'}{(2 \pi)^3}
        e^{i\vec k \cdot \vec x''} e^{i\vec k' \cdot \vec x'''}
        \langle \delta(\vec k, 0) \delta(\vec k', 0) \rangle\\
    &= b_e^2\int_0^\chi \ud \chi'' \int_0^{\chi'} \ud \chi'''
        g''g'''a''a'''
        \int_0^\infty\frac{\ud k}{2 \pi^2} k^2
        j_0 (k \sqrt{\chi''^2 + \chi'''^{^2} - \chi'' \chi''' \hat n \cdot \hat n'})
        P(k, 0)
\end{align}


Alternately, we can make the limber approximation (replacing
$a(\chi'')g(\chi'')$ and $a(\chi''')g(\chi''')$ with 
$a(\bar \chi)g(\bar \chi)$) and small angle approximations
(replacing the argument of the spherical Bessel with
$k\sqrt{\Delta \chi^2 + (\bar \chi (\hat n - \hat n'))^2}$).
\begin{align}
    w^{dd}(\hat n \cdot \hat n', \chi, \chi')
    &= b_e^2\int_0^\chi \ud \chi'' \int_0^{\chi'} \ud \chi'''
        (g(\bar \chi) a(\bar \chi))^2
        \int_0^\infty\frac{\ud k}{2 \pi^2} k^2
        j_0 (k \sqrt{\Delta \chi^2 + (\bar \chi (\hat n - \hat n'))^2})
        P(k, 0)\\
    &= b_e^2\int_0^{(\chi + \chi')/2} \ud \bar\chi 
        (g(\bar \chi) a(\bar \chi))^2
        \int_0^\infty\frac{\ud k}{2 \pi^2} k^2
        P(k, 0)
        \int_{-\infty}^{\infty} \ud \Delta\chi
        j_0 (k \sqrt{\Delta \chi^2 + (\bar \chi (\hat n - \hat n'))^2})\\
\end{align}

\end{comment}



\end{document}
