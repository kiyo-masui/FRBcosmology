\documentclass[twocolumn,nofootinbib,prl,floatfix]{revtex4-1}
\usepackage{graphicx}
\usepackage{amsmath}
\usepackage{amssymb}
\usepackage{bm}
\usepackage{color}
\usepackage{verbatim}
\usepackage{times}
\usepackage{afterpage}
\usepackage{epstopdf}
\usepackage{url}
\usepackage{hyperref}



% shortcuts
%\newcommand{\ud}{\,\mathrm{d}}
\newcommand{\bD}{\boldsymbol D}
\newcommand{\bC}{\boldsymbol C}
\newcommand{\bDelta}{\boldsymbol \Delta}
\newcommand{\Dgal}{D^{\rm gal}}
\newcommand{\Dne}{D^{\bar{n}_e}}
\newcommand{\Dde}{D^{\delta_e}}
\newcommand{\Dsrc}{D^{\rm src}}
\newcommand{\del}{\delta\!}
\newcommand{\calP}{{\cal P}}
%\newcommand{\ud}{\,\mathrm{d}}
\newcommand{\ud}{\,d}



% for draft comments.
\newcommand{\red}{\textcolor{red}}
\newcommand{\blue}{}
%\newcommand{\blue}{\textcolor{blue}}
\newcommand{\green}{\textcolor{green}}

% Command to consume spaces and newlines.  Used to PRL style sections.
\makeatletter
\def\ignorespacesandimplicitepars{%
  \begingroup
  \catcode13=10
  \@ifnextchar\relax
    {\endgroup}%
    {\endgroup}%
}


\renewcommand{\section}[1]{\emph{#1.}---\ignorespacesandimplicitepars}
\renewcommand{\subsection}[1]{}

%
%  These Macros are taken from the AAS TeX macro package version 4.0.
%  Include this file in your LaTeX source only if you are not using
%  the AAS TeX macro package and need to resolve the macro definitions
%  in the BibTeX entries returned by the ADS abstract service.
%
%  If you plan not to use this file to resolve the journal macros
%  rather than the whole AAS TeX macro package, you should save the
%  file as ``aas_macros.sty'' and then include it in your paper by
%  using a construct such as:
%	\documentstyle[11pt,aas_macros]{article}
%
%  For more information on the AASTeX macro package, please see the URL
%	http://www.aas.org/publications/aastex.html
%  For more information about ADS abstract server, please see the URL
%	http://adswww.harvard.edu/ads_abstracts.html
%

% Abbreviations for journals.  The object here is to provide authors
% with convenient shorthands for the most "popular" (often-cited)
% journals; the author can use these markup tags without being concerned
% about the exact form of the journal abbreviation, or its formatting.
% It is up to the keeper of the macros to make sure the macros expand
% to the proper text.  If macro package writers agree to all use the
% same TeX command name, authors only have to remember one thing, and
% the style file will take care of editorial preferences.  This also
% applies when a single journal decides to revamp its abbreviating
% scheme, as happened with the ApJ (Abt 1991).

\let\jnl@style=\rm
\def\ref@jnl#1{{\jnl@style#1}}

\def\aj{\ref@jnl{AJ}}                   % Astronomical Journal
\def\araa{\ref@jnl{ARA\&A}}             % Annual Review of Astron and Astrophys
\def\apj{\ref@jnl{ApJ}}                 % Astrophysical Journal
\def\apjl{\ref@jnl{ApJ}}                % Astrophysical Journal, Letters
\def\apjs{\ref@jnl{ApJS}}               % Astrophysical Journal, Supplement
\def\ao{\ref@jnl{Appl.~Opt.}}           % Applied Optics
\def\apss{\ref@jnl{Ap\&SS}}             % Astrophysics and Space Science
\def\aap{\ref@jnl{A\&A}}                % Astronomy and Astrophysics
\def\aapr{\ref@jnl{A\&A~Rev.}}          % Astronomy and Astrophysics Reviews
\def\aaps{\ref@jnl{A\&AS}}              % Astronomy and Astrophysics, Supplement
\def\azh{\ref@jnl{AZh}}                 % Astronomicheskii Zhurnal
\def\baas{\ref@jnl{BAAS}}               % Bulletin of the AAS
\def\jrasc{\ref@jnl{JRASC}}             % Journal of the RAS of Canada
\def\memras{\ref@jnl{MmRAS}}            % Memoirs of the RAS
\def\mnras{\ref@jnl{MNRAS}}             % Monthly Notices of the RAS
\def\pra{\ref@jnl{Phys.~Rev.~A}}        % Physical Review A: General Physics
\def\prb{\ref@jnl{Phys.~Rev.~B}}        % Physical Review B: Solid State
\def\prc{\ref@jnl{Phys.~Rev.~C}}        % Physical Review C
\def\prd{\ref@jnl{Phys.~Rev.~D}}        % Physical Review D
\def\pre{\ref@jnl{Phys.~Rev.~E}}        % Physical Review E
\def\prl{\ref@jnl{Phys.~Rev.~Lett.}}    % Physical Review Letters
\def\pasp{\ref@jnl{PASP}}               % Publications of the ASP
\def\pasj{\ref@jnl{PASJ}}               % Publications of the ASJ
\def\qjras{\ref@jnl{QJRAS}}             % Quarterly Journal of the RAS
\def\skytel{\ref@jnl{S\&T}}             % Sky and Telescope
\def\solphys{\ref@jnl{Sol.~Phys.}}      % Solar Physics
\def\sovast{\ref@jnl{Soviet~Ast.}}      % Soviet Astronomy
\def\ssr{\ref@jnl{Space~Sci.~Rev.}}     % Space Science Reviews
\def\zap{\ref@jnl{ZAp}}                 % Zeitschrift fuer Astrophysik
\def\nat{\ref@jnl{Nature}}              % Nature
\def\iaucirc{\ref@jnl{IAU~Circ.}}       % IAU Cirulars
\def\aplett{\ref@jnl{Astrophys.~Lett.}} % Astrophysics Letters
\def\apspr{\ref@jnl{Astrophys.~Space~Phys.~Res.}}
                % Astrophysics Space Physics Research
\def\bain{\ref@jnl{Bull.~Astron.~Inst.~Netherlands}} 
                % Bulletin Astronomical Institute of the Netherlands
\def\fcp{\ref@jnl{Fund.~Cosmic~Phys.}}  % Fundamental Cosmic Physics
\def\gca{\ref@jnl{Geochim.~Cosmochim.~Acta}}   % Geochimica Cosmochimica Acta
\def\grl{\ref@jnl{Geophys.~Res.~Lett.}} % Geophysics Research Letters
\def\jcp{\ref@jnl{J.~Chem.~Phys.}}      % Journal of Chemical Physics
\def\jgr{\ref@jnl{J.~Geophys.~Res.}}    % Journal of Geophysics Research
\def\jqsrt{\ref@jnl{J.~Quant.~Spec.~Radiat.~Transf.}}
                % Journal of Quantitiative Spectroscopy and Radiative Transfer
\def\memsai{\ref@jnl{Mem.~Soc.~Astron.~Italiana}}
                % Mem. Societa Astronomica Italiana
\def\nphysa{\ref@jnl{Nucl.~Phys.~A}}   % Nuclear Physics A
\def\physrep{\ref@jnl{Phys.~Rep.}}   % Physics Reports
\def\physscr{\ref@jnl{Phys.~Scr}}   % Physica Scripta
\def\planss{\ref@jnl{Planet.~Space~Sci.}}   % Planetary Space Science
\def\procspie{\ref@jnl{Proc.~SPIE}}   % Proceedings of the SPIE

\let\astap=\aap
\let\apjlett=\apjl
\let\apjsupp=\apjs
\let\applopt=\ao


\makeatother

\begin{document}

%\title{The large-scale structure of the Universe in dispersion measure space}
\title{Cross-Correlations Between Galaxies and Dispersion-Space Fast Radio
Bursts}

\author{Kiyoshi Wesley Masui}


\begin{abstract}

\end{abstract}

\maketitle
\section{Clustering in dispersion space}

The dispersion measure of a signal observed in some angular direction $\hat{n}$ and
originating from comoving radial distance $\chi$ is
\begin{equation}
    {\rm DM}(\hat{n}, \chi) = \int_0^{\chi} \ud\chi' a(\chi)^2
        n_e(\hat{n}\chi', \chi').
\end{equation}
Here
$n_e(\vec{x}, \chi)$ is the free electron density as a function of location and
conformal time. Note that we use $\chi$ as our radial distance and time
coordinate, as opposed to redshift. We model the cosmological 
electron density as containing a homogeneous part
and perturbations,
$n_e(\vec x, \chi) = \bar{n}_e(\chi) \left[1 + \delta_e(\vec x, \chi)\right]$.
The dispersion
measure is thus
\begin{equation}
    {\rm DM}(\hat n, \chi) = \int_0^\chi \ud\chi' a(\chi')^2 \bar{n}_e(\chi')
       \left[1 + \delta_e(\hat{n}\chi',\chi')\right].
\end{equation}

\emph{Dispersion space} is the three-dimensional coordinates,
$\vec x_s$, inferred from the dispersion measure assuming that the electrons are
homogeneous. This only affects the radial coordinate such that $\vec x_s = \hat
n \chi_s$, with $\chi_s$ defined by the equation
\begin{equation}
    {\rm DM}(\chi_s) = \int_0^{\chi_s} \ud\chi' a(\chi')^2 \bar{n}_e(\chi').
\end{equation}

Combining the above equations (keeping only terms first order in the density
perturbations), we find that
\begin{equation}
\frac{\ud \chi_s}{\ud \chi} = 1 + \delta_e(\hat n \chi, \chi),
\end{equation}
and thus
\begin{equation}
\chi_s - \chi = \int_0^\chi \ud \chi' \delta_e(\hat n \chi').
\end{equation}

We wish to relate the density of a tracer, $f$, measured in dispersion space to its
density in real space. We follow the derivation in \citet{1987MNRAS.227....1K}
of the
redshift-space distortions. Start by noting that the total number of tracers
in a volume element is the same in both spaces:
\begin{equation}
n_{fs}(\vec x_s) \ud^3\vec x_s = n_{f}(\vec x) \ud^3\vec x.
\end{equation}
We split the density into a homogeneous part plus perturbations,
\begin{equation}
\label{e:density}
    \bar{n}_{fs}(\chi_s)\left[ 1 + \delta_{fs}(\vec x_s)\right] \ud^3\vec x_s
    = \bar{n}_{f}(\chi)\left[ 1 + \delta_f(\vec x)\right] \ud^3\vec x.
\end{equation}
Averaged over the sky, $\langle \chi_s \rangle = \chi$, and thus the background
density should
be the same in dispersion space as in real space,
\begin{equation}
\bar{n}_{fs}(\chi) = \bar{n}_f(\chi).
\end{equation}
Therefore,
\begin{align}
\bar{n}_{fs}(\chi_s) 
    &= \bar{n}_{f}(\chi) + (\chi_s - \chi)\frac{\ud \bar{n}_f}{\ud \chi}\\
    &= \bar{n}_{f}(\chi)
       + \frac{\ud \bar{n}_f}{\ud \chi}\int_0^\chi \ud \chi' \delta_e(\hat n
       \chi').
       \label{e:nfs}
\end{align}

The Jacobian in spherical coordinates is
\begin{align}
\left| \frac{\ud^3\vec x}{\ud^3\vec x_s} \right|
    &= \frac{\ud \chi}{\ud \chi_s}\frac{\chi^2}{\chi_s^2}\\
%    &= (1 + \delta_e)^{-1}
%       \left(1 + \frac{\int_0^\chi \ud \chi' \delta_e(\hat n \chi')}
%                      {\chi}\right)^{-2}\\
    &\approx 1 - \delta_e
        - \frac{2}{\chi}\int_0^\chi \ud \chi' \delta_e(\hat n \chi').
        \label{e:jac}
\end{align}

Substituting Eqs.~\ref{e:nfs}~and~\ref{e:jac} into Eq.~\ref{e:density},
we obtain
\begin{equation}
\label{e:delta_s}
    \delta_{fs} = \delta_f - \delta_e
    - \left(\frac{1}{\bar{n}_f}\frac{\ud \bar{n}_f}{\ud \chi}
    + \frac{2}{\chi} \right)
        \int_0^\chi \ud \chi' \delta_e(\hat n \chi').
\end{equation}
In the above equation, the $-\delta_e$ term is most analogous to the Kaiser
redshift-space distortions.  It is a dilution of tracers in dispersion space due to an
excess of electrons between the tracers. However we note that this term is isotropic
in contrast to the Kaiser term.  This is because any wave vector electron perturbation
causes dispersion-space distortions, whereas the radial velocities that cause
redshift-space distortions are only sourced by perturbations with radial wave
vectors.

The $\frac{1}{\bar{n}_f}\frac{\ud \bar{n}_f}{\ud \chi}$ term arises because the
misinterpretation of the radial distance causes the observed tracer density to
be compared to the wrong background density. The $\frac{2}{\chi}$ term is
caused by a misinterpretation of angular distances when the radial distance is
mismeasured.  In both cases, analogous terms are, in principle, present in 
redshift space but are
negligible. Because radial velocities are only sourced by modes with a large
radial wave vector, there is near perfect cancellation along the line of sight,
and thus there is very little net error in the radial distance.

For brevity in the following sections, we define the coefficient of the integral
term as
\begin{equation}
    \label{e:A}
    A(\chi) \equiv \frac{1}{\bar{n}_f}\frac{\ud \bar{n}_f}{\ud \chi}
    + \frac{2}{\chi}.
\end{equation}


\section{Cross-correlation}

Large-scale structure is usually studied through its two-point statistics, most
commonly the power spectrum, $P(k)$.
Unlike the redshift-space distortions, Eq.~\ref{e:delta_s} does not have a
simple form in harmonic space. The equation's third term couples harmonic
modes,
and thus the two-point statistics cannot be phrased as a simple power spectrum.
We will instead use $C^{gs}_\ell(\chi,\chi')$, which is
the cross-correlation angular power spectrum of the dispersion-space overdensity,
on shells at $\chi$ and $\chi'$:
\begin{align}
    &C^{gs}_\ell(\chi, \chi') =
    \int\ud\Omega\ud\Omega' Y_{\ell m}(\hat n) Y^*_{\ell m}(\hat n')
    \langle \delta_{g}(\hat n \chi) \delta_{fs}(\hat n' \chi')
        \rangle.
\end{align}
The first two terms in Eq.~\ref{e:delta_s} are stationary. For stationary,
isotropic tracers $x$ and $y$, we have 
$\langle \delta_x(\vec k, \chi) \delta_y(\vec k', \chi) \rangle = (2\pi)^3
\delta^3(\vec k - \vec k') P_{xy}(k, \chi)$.  If, for a moment, we ignore
structure evolution, the angular cross-power spectrum of such tracers is
\begin{align}
    %\delta_{\ell\ell'}\delta_{mm'}C^{xy}_\ell(\chi, \chi')
    %&= \int\ud\Omega\ud\Omega'Y_{\ell m}(\hat n) Y_{\ell' m'}(\hat n')
    %    \langle \delta_x(\hat n \chi) \delta_y(\hat n' \chi') \rangle,
    %    \\
    C^{xy}_\ell(\chi,\chi')
    &= \frac{2}{\pi}
\int_0^\infty\ud k k^2 j_\ell(k\chi) j_{\ell}(k\chi')P_{xy}(k).
\end{align}
In reality the power spectrum evolves on the order of a Hubble time.
The angular cross correlations will be very small unless $\chi$ and $\chi'$ are
within a few correlation lengths of one another, roughly a hundred
megaparsecs.  The evolution of the power spectrum is negligible over these time
differences, which leads to a straightforward way to include the evolution:
\begin{align}
    % I've tuned the alignment of the domain (using &)
    % here to get the equation number and
    % domain on the same line. Not elegant.
    C^{xy}_\ell(\chi,\chi') \approx&
        \frac{2}{\pi}
        \int_0^\infty\ud k k^2
        j_\ell(k\chi) j_{\ell}(k\chi')
        P_{xy}(k,(\chi + \chi')/2),\nonumber \\
    &|\chi - \chi'| \ll 1/aH.
\end{align}

The third term in Eq.~\ref{e:delta_s} is not stationary but is an integral
over the stationary field $\delta_e$.
Define $\delta_d$ as
\begin{equation}
    \delta_d(\hat n \chi) \equiv \int_0^\chi \ud \chi' \delta_e(\hat n \chi').
\end{equation}
It is straightforward to show that
\begin{widetext}
\begin{equation}
C^{dd}_\ell(\chi,\chi')
    =
    \frac{2}{\pi}
    \int_0^\chi\ud\chi''
    \int_0^{\chi'}\ud\chi'''
    \int_0^\infty\ud k k^2 j_\ell(k\chi'') j_{\ell}(k\chi''')
    P_{ee}(k, (\chi''+\chi''')/2).
\end{equation}

Finally, $C^{gs}_\ell$ will contain cross terms between the stationary terms
and the integral terms.  These will have the form
\begin{equation}
C^{dx}_\ell(\chi,\chi')
    =
    \frac{2}{\pi}
    \int_0^\chi\ud\chi''
    \int_0^\infty\ud k k^2 
    j_\ell(k\chi') j_{\ell}(k\chi'')P_{ex}(k, (\chi' + \chi'')/2).
\end{equation}

Assembling all of these expressions with the proper coefficients, we have
\begin{align}
    \label{e:Clss}
C^{gs}_\ell(\chi,\chi') = &~
    \frac{2}{\pi}
    \int_0^\infty\ud k k^2
    j_\ell(k\chi) j_{\ell}(k\chi')
    P_{[gf - ge]}(k, (\chi + \chi')/2)
    %\left[
    %P_{ff}(k, \bar\chi)
    %+ P_{ee}(k, \bar\chi)
    %- 2P_{fe}(k, \bar\chi)
    %\right]
    \nonumber\\
    & +
    \frac{2}{\pi}
    %\left(\frac{1}{\bar{n}_f}\frac{\ud \bar{n}_f}{\ud \chi'}
    %+ \frac{2}{\chi'} \right)
    A(\chi')
    \int_0^{\chi'}\ud\chi''
    \int_0^\infty\ud k k^2 
    j_\ell(k\chi) j_{\ell}(k\chi'')
    P_{[-ge]}(k, (\chi + \chi'')/2).
    %\left[ 
    %P_{ee}(k, (\chi + \chi'')/2)
    %- P_{fe}(k, (\chi + \chi'')/2) 
    %\right].
\end{align}
\end{widetext}
Here, expressions such as $P_{[ff + ee - 2ef]}$ are shorthand for $P_{ff} +
P_{ee} - 2P_{ef}$.

Equation~\ref{e:Clss} can be simplified substantially by adopting the small
angle and Limber approximations \citep{1953ApJ...117..134L,
1992ApJ...388..272K, 1998ApJ...498...26K}.  The small angle
approximation eliminates the $k$ integral over spherical Bessel functions, replacing it
with a Fourier transform, and is valid for $\ell \gg 1$. The Limber
approximation assumes that only modes with a small radial component of their wave
vector contribute to the radial integrals and is valid if the power spectra
evolve slowly compared to the correlation length \citep{2008PhRvD..78l3506L}
(which has already been
assumed). With these approximations, we have
\begin{align}
    &C^{gs}_\ell(\chi, \chi') \approx
    \nonumber \\ & \quad
    \frac{1}{\bar\chi^2}
        \int_{-\infty}^\infty\frac{\ud k_\parallel}{(2 \pi)} 
        e^{i k_\parallel (\chi - \chi')}
        P_{[gf - ge]}
        (\sqrt{k_\parallel^2 + \nu^2/\bar\chi^2}, \bar\chi)
    \nonumber \\ & \quad +
    H(\chi' - \chi)
    \frac{A(\chi')}{\chi^2}
    P_{[- ge]}(\nu/\chi, \chi),
    \label{e:Clss_limber}
\end{align}
where $\nu\equiv\ell + 1/2$, $\chi_{\min} \equiv \min(\chi, \chi')$,
$\chi_{\max} \equiv \max(\chi, \chi')$, and $H$ is the Heaviside step function.
We have found that these approximations are accurate to within $3\%$ at
$\ell \gtrsim 10$
and use this form for the remainder of the Letter. 

One thing that should be true is that the disperion-space distortions
should no effect
on observations that do not use DM information. Lets say we did not bin the
FRBs by dispersion measure but just took the angular cross-power spectrum of the
unbinned FRBs with the galaxies (which may still be binned), $C^{gf}_\ell(\chi)$.
The number of FRBs per unit comoving distance is proportional to
$\chi^2 \bar{n}_{f}(\chi)$, and so
\begin{equation}
    C^{gf}_\ell(\chi) \propto \int \ud\chi' {\chi'}^2 \bar{n}_{f}(\chi')
    C^{gs}_\ell(\chi, \chi').
\end{equation}
Substituting Equation~\ref{e:Clss}, we find that integrating the cross term by
parts gives exact expression required to cancel the $P_{ge}$ part of the local
term and everything is miraculously consistent.


\bibliography{refs}

\end{document}


