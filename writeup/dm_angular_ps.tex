\documentclass[onecolumn,prd,noshowpacs,nofootinbib,amsmath,amssymb]{revtex4}
\usepackage{graphicx,epsfig,psfrag,bm,amssymb}
\usepackage{dcolumn}
\usepackage{bm}
\usepackage{mciteplus}
\usepackage{color}
\usepackage{mathrsfs,amsfonts,hepunits, color}
\usepackage{mciteplus}
\usepackage{tikz}
\usepackage{slashed}
\usepackage{verbatim}


% shortcuts
%\newcommand{\ud}{\,\mathrm{d}}
\newcommand{\bD}{\boldsymbol D}
\newcommand{\bC}{\boldsymbol C}
\newcommand{\bDelta}{\boldsymbol \Delta}
\newcommand{\Dgal}{D^{\rm gal}}
\newcommand{\Dne}{D^{\bar{n}_e}}
\newcommand{\Dde}{D^{\delta_e}}
\newcommand{\Dsrc}{D^{\rm src}}
\newcommand{\del}{\delta\!}
\newcommand{\calP}{{\cal P}}
\newcommand{\ud}{\,\mathrm{d}}



% for draft comments.
\newcommand{\red}{\textcolor{red}}
\newcommand{\blue}{\textcolor{blue}}
\newcommand{\green}{\textcolor{green}}




\begin{document}

\title{Fast radio burst cosmology: dispersion measure angular power spectrum}

\author{Kris Sigurdson}
 \affiliation{Department of Physics and Astronomy, University of British 
Columbia, Vancouver, BC, Canada, V6T 1Z1}

\author{Kiyoshi Wesley Masui}
 \affiliation{Department of Physics and Astronomy, University of British 
Columbia, Vancouver, BC, Canada, V6T 1Z1}
 \affiliation{Canadian Institute for Advanced Research, CIFAR Program in
Cosmology and Gravity, Toronto, ON, Canada, M5G 1Z8}

\begin{abstract}

\end{abstract}

\maketitle


\section{Introduction}

Several years ago an unusual radio event was discovered at the Parks radio telescope while searching for millisecond pulsars.  In a direction out of the plane of the Galaxy an event was discovered with an unusually large dispersion-measure, implying a column of electrons much large than expected in the Galaxy.

In this \emph{ Letter} we consider the implications of of a cosmological population of fast radio bursts (FRBs). 
If FRBs originate at cosmological distances they provide a new window on the Universe. The key insight into the potential power of FRBs is that, while it is interesting and helpful to know the redshift of the burst, it is not necessary to do interesting studies of cosmology.

In this paper we consider only the simplest statistic that captures some of
the cosmological information present in an FRB survey.  This is the dispersion
measure angular power-spectrum, which is conceptually in vary close analogy to
the galaxy weak-lensing shear angular power spectrum. We start with a
population of FRB events (the analogue of lensed galaxies) whose redshifts are
unknown.  In the first instance we ignore the intrinsic clustering of these
background sources.  The principal observable is the dispersion measure
(shear) which is a line-of-sight integral of the intervening large-scale
structure.  This is contaminated by a noise term (analogue of intrinsic
ellipticity) which is the dispersion
measure originating from the host halo as well as the dispersion measure
originating from the spacial mean electron density coupled with the unknown
redshift.  In both cases there is a systematic effect originating from the
intrinsic clustering of source objects, which is that pairs of sources with
small angular separation are likely associated with the same large-scale
structure and thus have similar redshifts.  This induces correlations in the
noise tern associated with the unknown distance through the mean electron
density. The analogue in galaxy lensing is intrinsic alignments, however while
intrinsic alignments are a non-linear effect, the redshift clustering term occurs
at linear order.


\section{Preliminaries}

% Dodelson uses \chi not \eta... should we used \chi?

The dispersion measure of a FRB observed in some angular direction $\hat{n}$ and
originating from redshift $z$ is
\begin{equation}
    {\rm DM}(\hat{n}, z) = \int_0^{\eta(z)} d\eta'
        \frac{n_e(\hat{n}, z')}{(1+z')^2}.
\end{equation}
Here, $\eta$ is the conformal distance as a function of redshift, and
$n_e(\hat{n}, z)$ is the three dimensional electron density.  We choose to
separate $n_e$ into four components as follows:
\begin{equation}
\label{e:ne_sep}
n_e(\hat{n}, z) = n_e^{\rm gal}(\hat{n}, z) + \bar{n}_e(z) +
    \bar{n}_e(z)\delta_e(\hat{n}, z) + n_e^{\rm src}(\hat{n}, z).
\end{equation}
Here $n_e^{\rm gal}(\hat{n}, z)$ is the contribution to the electron density
from the Milky Way galaxy and other local structures and only contributes at
low redshift. $\bar{n}_e(z)$ is the mean electron density is and related to the
current day electron density by $\bar{n}_e = \bar{n}_{e0} (1+z)^3$. 
$\delta_e(z,\hat{n})$ is the large-scale fractional over-density of electrons.
$n_e^{\rm src}(\hat{n}, z)$ are the electrons associated with the FRB source,
which only contributes as $z \approx z^{\rm src}$ and accounts for two effects.
First the FRB itself
may have some intrinsic dispersion measure due to the environment of the
progenitor.  Second, it is unlikely that FRB events are uniformly distributed
throughout the Universe and they presumably trace matter in some way. When
considering statics of FRB dispersion measures it is necessary to account for
this bias.

We define the dispersion-measure distance (or dispersion distance),
the equivalent distance required to
produce a given amount of dispersion in an observed radio signal in the
Universe today, as $ D = {{\rm DM}}/{\bar{n}_{e0}}$
We separate the dispersion distance into the contributions from each
term in Equation~\ref{e:ne_sep}:
\begin{equation}
\label{e:D_sep}
D = \Dgal + \Dne + \Dde + \Dsrc
\end{equation}
For a population of FRBs originating at cosmological distances, $\Dgal$
is only a single function of $\hat{n}$ which can presumably be well
measured.  In addition, this function could in principal be well characterized
by independent data such as observations of pulsars.  As such, we will
here-after we ignore this term.

In addition, we define the mean subtracted values of the of the variouse dispersion
measure quantites, $\del D = D - \langle D \rangle$. Here the angular brackets represent
the ensemble average.

\red{Introduce the redshift distribution function, assumption that FRB's trace
LSS, bias, etc.}

\red{Convensions: we use subscripts $i$, $j$, etc.~on coordinate-like
    quantities to label observed FRB events. Thus $\vec x_i$ and $\hat{n}_i$
    are respectively the 3D coordinates of and unit vector pointing to event $i$.
    This should not be confused with superscripts, denoting the
component of a vector, such that $x^1$ is the first componenent of vector $\vec
x$.  }

\red{$\calP$ is a probability, while $p$ is a PDF. $\calP(x < X < x + \ud x) =
p(x) \ud x$ always.}


\section{Dispersion Measure angular correlations}

The principle observable which we would like to consider is the correlations
between the dispersion measures of observed fast radio bursts as a function of
their angular separation.  For observed FRB events labeled by the indices $i$
and $j$, and assuming $i \neq j$, we define the dispersion measure angular
correlation function as
\begin{equation}
    w(\hat{n}_i\cdot\hat{n}_j) \equiv \langle \del D_i \del D_j \rangle.
\end{equation}
A move convienient quantity to work with is the corresponding angular
power spectrum
\begin{equation}
w(\hat{n}_i \cdot\hat{n}_j) = \sum_l \frac{2l+1}{4 \pi} C_l
P_l(\hat{n}_i\cdot\hat{n}_j).
\end{equation}

We have identified two effects that dominate the angular
correlations and which depend on the large-scale structure of the Universe and
thus cosmology.  The first effect is that the signals from two FRB events may
propagate through shared structures along the line of sight.  Thus correlations
in the election overdensities along the line of sight will correlate the 
$\Dde$.  The second effect is that due to the intrisic clustering of source
events, a pair of FRBs that are nearby in the angular coordinates are more
likely to also be nearby in the line of sight coordinate, thus causing
correations in $\Dne$.  In the remainder of this section we will calculate the
contributions from each of these effects.

We assume that $D^{\rm src}_i$ are uncorrelated. Again this may not
be strictly true due to clustering of FRB sources and environmental effects
and this may bias results.  In addition $\Dsrc$ and $\Dne$ may be correlated due
to redshift evolution of source events.


\subsection{Correlations in $\Dde$}

\begin{equation}
C_l^{zz'}=\frac{2}{\pi} \int_0^{\infty}k^2 dk P_{ee}(k) \int_0^{\eta^z}\frac{d\eta}{a(\eta)} \int_0^{\eta^{z'}}\frac{d\eta'}{a(\eta')}  j_l(k\eta) j_l(k\eta')
\end{equation}
\begin{equation}
C_l^{\Dde \Dde} = \int d\eta^z \int d\eta^{z'} p(\eta^z)p(\eta^{z'}) C_l^{z z'}
\end{equation}


\subsection{Correlations in $\Dne$}

The joint probablility of observing a pair of FRB events at 3D coordinates
$\vec x_i$ and $\vec x_j$ in infintessimal volumes $\ud^3\vec x_i$ and
$\ud^3\vec x_j$ is
\begin{equation}
    \calP(\vec x_i, \vec x_j) = p(\eta_i) p(\eta_j) \left[ 1 + b(z_i)
    b(z_j) \xi(\vec x_i - \vec x_j) \right] \frac{\ud^3\vec x_i}{\eta_i^2}
    \frac{\ud^3\vec x_j}{\eta_j^2}.
\end{equation}
The factors of $1/\eta^2$ are nessisary to achive the proper volume weighting,
a fact that can be seen by taking the limit of $\xi \to 0$ and integrating of
the angular coordinates. Noting that $p(\eta_i)$ has units 
\red{$({\rm MPc}\,{\rm rad}^2)^{-1}$ (conflicts with being normalized)}.

An observed event has known angular coordinates, and we are interested in
correlations in $\Dne$ which is directly related to the line of sight
coordinate.  Fixing the angular coordinates:
\begin{equation}
    \calP(\eta_i, \eta_j | \hat{n}_i, \hat{n}_i) = N(\hat{n}_i, \hat{n}_i)
    p(\eta_i)p(\eta_j) \left[ 1 + b(z_i)
    b(z_j) \xi(\vec x_i - \vec x_j) \right] \ud \eta_i \ud \eta_j,
\end{equation}
where $N(\hat{n}_i, \hat{n}_i)$ is a normalizing factor which ensures the
expression integrates to unity over $\eta_i$ and $\eta_j$. The factors of
$1/\eta^2$ have been canceled when converting $\ud^3 \vec x = \eta^2 \ud \eta \ud^2
\vec \theta$. If the second term in brackets is small, then the normalizations
$N$ is 1 (since the distributions $p{\eta_i}$ are already normalized. 
This expression is
only valid if $i \ne j$.
Transforming to $\Dne$ as our distance measure, we obtain
\begin{align}
\calP(\Dne_i, \Dne_j | \hat{n}_i, \hat{n}_i)
    &= p(\eta_i)p(\eta_j) \left[ 1 + b(z_i)
        b(z_j) \xi(\vec x_i - \vec x_j) \right]
        \frac{\ud \eta_i}{\ud \Dne_i} \frac{\ud \eta_j}{\ud \Dne_j}
        \ud \Dne_i\ud \Dne_j
        \\
    &= p(\eta_i)p(\eta_j) \left[ 1 + b(z_i)
        b(z_j) \xi(\vec x_i - \vec x_j) \right]
        \frac{\ud \Dne_i}{1 + z_i}\frac{\ud \Dne_j}{1 + z_j}
\end{align}
Finally, taking the second moment and subtracting off the product of the first
moments:
\begin{equation}
\langle \Dne_i \Dne_j \rangle
    - \langle \Dne_i \rangle \langle \Dne_j \rangle =
    \int \Dne_i \Dne_j
    \frac{p(\eta_i)}{1 + z_i}\frac{p(\eta_j)}{1 + z_j}
    b(z_i) b(z_j) \xi(\vec x_i - \vec x_j)
        \ud \Dne_i\ud \Dne_j
\end{equation}
And thus we have:
\begin{equation}
    w^{\bar{n}_e \bar{n}_e}(\hat{n}_i\cdot\hat{n}_j)
        =\int \Dne_i \Dne_j
    \frac{p(\eta_i)}{1 + z_i}\frac{p(\eta_j)}{1 + z_j}
    b(z_i) b(z_j) \xi(\eta_i \hat{n}_i - \eta_j \hat{n}_j)
        \ud \Dne_i\ud \Dne_j
\end{equation}
Where $\eta$ and $z$ are understood to be functions of $\Dne$.



\section{Modeling angular correlations}


\section{Estimating angular correlations}

We observe a set of observed dispersion measures
$\boldsymbol \Delta$ (whose members are $\delta D_i$)
with known angular positions, $\hat n_i$, but with unknown
redshifts, $z_i$.  The Covariance matrix is
${\boldsymbol C} \equiv
\langle \boldsymbol \Delta \boldsymbol \Delta^T \rangle$.
The off diagonal elements depend on cosmology
$C_{ij} = w(\hat{n}_i\cdot\hat{n}_j)$ while the diagonal part is dominated by noise.

\subsection{Noise}

The noise will be dominated by contributions from $\Dsrc$, $\Dne$ and $\Dde$.

\begin{align}
    \label{e:sig_nebar}
    \langle \delta D^{\bar{n}_e}_i \delta D^{\bar{n}_e}_i \rangle
        &= \left[ \int_0^\infty \ud D^{\bar{n}_e} (D^{\bar{n}_e})^2
            p(D^{\bar{n}_e})
            - \left( \int_0^\infty \ud D^{\bar{n}_e} D^{\bar{n}_e}
        p(D^{\bar{n}_e}) \right)^2 \right]\\
        &\equiv  \sigma_{\bar{n}_e}^2
\end{align}

\begin{equation}
    \label{e:sig_src}
    \langle \delta D^{\rm src}_i \delta D^{\rm src}_i \rangle
    \equiv \sigma^2_{\rm src}.
\end{equation}

\begin{equation}
    \label{e:sig_delta}
    \langle \delta \Dde_i \delta \Dde_i \rangle
		= \int d\eta^z \int d\eta^{z'} \delta(\eta^z - \eta^{z'}) 
		p(\eta^{z}) w^{z z'}(0)
\end{equation}

\subsection{Estimators}

Here we sketch how the dispersion mearure angular power-spectrum, $C_l$
could be estimated from a sample of observed dispersion-measures.

The covariance matrix of the sample is:

\begin{align}
    C_{ij} \equiv& \langle \delta D_i \delta D_j \rangle \\
    =&
        \begin{cases}
            \sum_l \frac{2l+1}{4 \pi} C_l P_l(\hat{n}_i\cdot\hat{n}_j)
                &  \text{if } (i \neq j) \\
            \sigma_{\bar{n}_e}^2 + \sigma^2_{\rm src}
                + \textrm{(the $w$ term)}
                &  \text{if } (i = j)
        \end{cases}
\end{align}

We see that the off diagonal terms depend on the angular power-spectrum of
interest while the diagonal is contaminiated by the unknown parameter
$\sigma_{\rm src}^2$. This is again in analogue to weak lensing where the
correlations between galaxy elipticty are caused primarily by cosmic leansing shear
while an individual galaxies elipticity is dominated the galaxy's intrinsic
elipticity. To remove the information in the diagonal, we define a
modified covariance matrix derivative such that:
\begin{equation}
    \tilde{C}_{ij,\alpha} \equiv
\begin{cases}
    C_{ij,\alpha} & \text{if } i \neq j,\\
0 & \text{if } i = j
\end{cases}.
\end{equation}

We will parameterize the angular power spectrum using band powers, such that:
\begin{equation}
    C_l \approx c_\alpha \quad \textrm{for $l^{min}_\alpha \leq l <
        l^{min}_{\alpha +1}$}
\end{equation}
giving
\begin{equation}
    C_{ij, \alpha} = \sum_{l = l^{min}_\alpha}^{l^{min}_{\alpha +1} - 1}
        \frac{2l+1}{4 \pi} P_l(\hat{n}_i\cdot\hat{n}_j)
        \quad\textrm{for $i \neq j$}.
\end{equation}

The estimator is then

\begin{equation}
    \hat{c}_\alpha = \frac{\tilde F^{-1}_{\alpha\beta}}{2} 
        \left[ \bDelta^T \bC^{-1} \tilde\bC_{,\beta} \bC^{-1} \bDelta \right],
\end{equation}
with
\begin{equation}
    \tilde F_{\alpha \beta} = \frac{1}{2} \mathrm{Tr} 
        \left[ \tilde \bC_{,\alpha} \bC^{-1} \tilde\bC_{,\beta} \bC^{-1}
        \right].
\end{equation}



\subsection{Sensitivity Estimates}

In this section we present a simplified calculation for the sensitivity of the
above estimator which is valid for sample of observed
FRBs distributed roughly uniformly over some fraction of the sky.

The error bar on the redshift averaged dispersion measure angular power spectrum
is
\begin{equation}
    \sigma_{C_l} = \sqrt{\frac{2}{(2 l + 1) f_{\rm sky}}} (C_l + C^N_l),
\end{equation}
where the noise power spectrum is
\begin{equation}
    C^N_l = \frac{4 \pi \sigma_D^2 \sqrt{f_{\rm sky}}}{N_{\rm FRB}}.
\end{equation}
Here $N_{\rm FRB}$ is the number of dispersion measures measured from fast
radio bursts. \red{this depends on the off diagonals of the covariance being
small compared to the diagonals}.
%\red{I'm only 80\% sure about the factors of $f_{\rm sky}$}.

$\sigma_D^2$ is the variance of the dispersion measures.  This would in
practise be measured directly from the observed set, but in terms of
theoretical parameters is
\begin{equation}
\sigma_D^2 = \sigma_{\bar{n}_e}^2 + \sigma^2_{\rm src} + \int \ud z \ud z'
\delta(z - z') p(z) w^{z z'}(1),
\end{equation}
where the first two terms are defined above in equations \ref{e:sig_nebar}
through
\ref{e:sig_src}. The third term is likely
negligible and could also be written as
\begin{equation}
    \int \ud z \ud z' \delta(z - z') p(z) w^{z z'}(1) = \int \ud z \ud z'
    \delta(z - z') p(z) \sum_{l} \frac{2l+1}{4\pi} C_l^{zz}.
\end{equation}



\section{Extensions to 3D}

In the above discussions we have focused on the dispersion measure angular
power spectrum due to its conceptual and mathematical simplicity.
However, it is unlikely that this 2D statistic captures all or even most of the
LSS information in a survey.  In this section we discuss possible extensions to
3D statistics that might capture a larger portion of the total available
information.

When discussing the DM angular power spectrum, we treated the dispersion
measure as the \emph{dependant} variable and considered statics. However,
$\Dne$ is in one to one relation to source redshift, and as such the dispersion
measure can be used as a proxy for radial distance.  As such we promote the
dispersion measure to be the \emph{independent} variable and consider the
statistics of the 3D number density field of FRB source events.
If $\Dne$ were the only contribution to the dispersion measure then, assuming
the FRB source events are a biased tracer of matter structure on large scales,
this probe would be broadly similar to galaxy redshift surveys.  Variations in
$\Dsrc$ produce a 'redshift smearing', limiting the effective resolution in the
radial direction. The contributions to the dispersion measure from the electron
over densities, $\Dde$, produce a variety of cosmological signals which we
describe below.


\subsection{Dispersion measure space distortions}

Perturbations in the electron column density due to the intervening large-scale
structure of free electrons results in a mis-estimation of radial distance
from dispersion measure.  We've identified two effects through which this
mis-estimation can cause \emph{unclustered} FRB source events to appear
clustered, which we call the 'dilution effect' and the 'scattering effect'.

In the dilution effect, FRB events embedded in electron over densities have
additional elections (relative to the mean electron density) between them and
thus appear to be more separated along the line of sight and less
clustered.  If both the free electrons and the FRB source events trace the
large-scale structure, this effect is anti correlated with the intrinsic
clustering of the source events, although highly anisotropic.

The scattering effect occurs when perturbations in the foreground free electrons
cause the radial distances to background source events to be mis-estimated. 
When coupled with a non-constant radial distribution
function, $p(\eta)$, this can cause FRB source events to
appear clustered. For instance, if there is an electron over-density in some
direction background FRB events in that direction will be interpreted to be at
a greater radial distance than they are. If $\ud p(\eta) /\ud \eta)$ is
negative, there will be an excess of source events assigned to a given radial
distance behind the electron over density which can be interpreted as apparent
clustering of intrinsically unclustered source events.

Collectively we dub these effects 'dispersion measure space distortions' due
to its broad
similarity to redshift space distortions. In redshift space distortions peculiar
velocities of tracers induce a mis-estimation of radial distance using redshift.
The convergence of these mis-estimations, proportional to the convergence of the
peculiar velocities causes unclustered tracers to appear clustered along the line
of sight. Since the velocity convergence and the over density are perfectly
correlated in linear theory this manifests as an enhancement of line of sight
clustering.

By moving to 3D we fully exploit the redshift information inherent in $\Dne$ as
opposed to treating it largely as a noise term as in our above discussion on
the dispersion measure angular power spectrum. As such we expect such a
treatment would greatly increase the amount of information extracted from an
FRB survey, although we make no effort to quantify this statement.

In addition we expect that in such a treatment the degeneracy between FRB
source clustering and free electron clustering would be significantly less
severe. This is because in 3D, the intrinsic source clustering should be perfectly
isotropic while the electron clustering induces apparent source clustering
only in the radial direction and should thus be highly anisotropic.


\subsection{Four point statistics}

In the above dispersion measure space distortions discussion, we discussed how
scattering and dilution can cause unclustered FRB source events to appear to
be clustered in three dimensions.  These effects also have an effect on the
clustered source events. By scattering or diluting the radial distance of
clustered source events, the free electron density field modulates the
statistics of the source clustering. The two point statistics become
non-stationary, which can be measure in the four-point statistics of the source
event density as a non-gaussianity.


\section{Conclusions}




\end{document}
