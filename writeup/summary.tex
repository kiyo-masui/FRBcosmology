\documentclass{article}

\title{Mysterious radio flashes to be used to trace Universe's largest
structures}

\date{\today}

\author{Kiyoshi Wesley Masui and Kris Sigurdson}

\begin{document}

\maketitle

One of the most challenging aspects of astronomy is knowing distances to far
off objects. Newly discovered fast radio bursts---originating from across the
Universe---provide distance measurements that will enable the largest
structures of the cosmos to be studied in three dimensions. Traditionally,
cosmological distances are measured using Doppler shifts caused by the Hubble
expansion of the Universe. Distortions to brief radio signals can be used to
measure the total amount material between earth and the source, permitting
another way to measure distance. However this process isn't perfect.
Intervening clumps of matter can distort the distance measure, a fact that can
be exploited to glean yet more information about the Universe's structure.
Wide-field radio telescopes are expecting to detect thousands of these radio
bursts, opening a new window for studying the cosmos.


\end{document}
