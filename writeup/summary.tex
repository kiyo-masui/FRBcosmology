\documentclass{article}

\title{Mysterious radio flashes to be used to trace Universe's largest
structures}

\date{\today}

\author{Kiyoshi Wesley Masui and Kris Sigurdson}

\begin{document}

\maketitle

One of the most challenging aspects of astronomy is determining distances to
far-off objects. Newly discovered fast radio bursts---thought to originate from
the far reaches of the Universe---provide distance measurements that will
enable the largest structures of the cosmos to be studied in three dimensions.
Traditionally cosmological distances are obtained using Doppler shifts, caused
by the Hubble expansion of the Universe. In contrast, distortions to fast radio
burst signals can be used to measure the total amount material between earth
and the burst's source, permitting another way to determine distance.  However
this process isn't perfect, as intervening clumps of excess matter can distort the
measured distance. This fact can be exploited to glean yet more
information through these intervening structures.  Wide-field radio telescopes
are expecting to detect thousands of these bursts, opening a new window for
studying the cosmos.


\end{document}
