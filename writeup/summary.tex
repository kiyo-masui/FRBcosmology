\documentclass{article}

\title{Mysterious radio flashes could trace Universe's largest
structures}

\date{\today}

\author{Kiyoshi Wesley Masui and Kris Sigurdson}

\begin{document}

\maketitle

One of the most challenging aspects of astronomy is determining distances to
far-off objects. Newly discovered fast radio bursts---thought to originate from
the far reaches of the Universe---provide distance measurements that may
enable the largest structures of the cosmos to be studied in three dimensions.
Cosmological distances are traditionally determined using Doppler shifts, caused
by the Hubble expansion of the Universe. In contrast, delays to fast radio
burst signals can be used to tally the total amount of material between earth
and the burst's source, permitting another way to determine distance.
As intervening clumps of cosmic matter can add extra delay, these
measured distances don't exactly match the actual distances.
However, these distortions reveal information about the intervening
structure.
Wide-field radio telescopes could detect thousands of these bursts, opening a
new window for studying the cosmos.


\end{document}
